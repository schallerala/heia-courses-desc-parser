% id: 3
\gradedCourse{Human-Computer Interaction}{4}{5}{Interface Homme-Machine 1}{french}{Sandy~Ingram}
Course content included:
\begin{itemize}
    \item Putting into practice the elements seen during programming: classes, abstract classes, interfaces, internal classes, enum, genericity, annotations, lambda expressions, method references, etc.
    \item Technical aspects related to the use of JavaFX (Stage, scene graphs, containers, UI components, properties, binding, event management, ...). Principles of event-driven programming.
    \item Using the SceneBuilder tool to generate graphical interfaces in declarative mode (XML / FXML format).
    \item Splitting of an application according to the Model-View-Controller (MVC) architecture.
    \item Drop-down and context-sensitive menus. Creation of keyboard shortcuts.
    \item Creation and use of general and specialized dialog boxes.
    \item Deployment of an application with external resources (images, sound, ...).
    \item General principles related to human-machine interaction, usability, ergonomic criteria and human factors.
    \item Ergonomic assessments: techniques and implementation.
    \item Elements related to visualization (icons, highlighting techniques, layout and alignment, screen templates, gaze path, layout, use of colors, definitions and typographical rules, ...).
\end{itemize}