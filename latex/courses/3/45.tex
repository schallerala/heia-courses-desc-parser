% id: 45
\gradedCourse{Elective: Macine Learning}{2}{4.9}{Chapitre spécialisé: Machine Learning Applications}{french}{Jean Hennebert, Elena Mugellini}
Introduction:
\begin{itemize}
    \item Machine Learning is a branch of Artificial Intelligence that studies so-called machine learning algorithms. These algorithms are capable, using examples, of solving complex problems that would be difficult to solve using traditional approaches. Machine Learning is used today in several fields: prediction (stock market evolution, weather), classification (gesture recognition, speech recognition, image recognition), verification/detection (biometric authentication), data mining (clustering on complex data).
\end{itemize}
Method:
\begin{itemize}
    \item The course combines theoretical and practical sessions. Of particular importance is given to practical sessions, which will be done through guided classroom exercises (including the use of the KNIME and WEKA tools) and through mini-projects carried out in groups of 2-3 people.
\end{itemize}
Part I - Basic principles (20\%):
\begin{itemize}
    \item Definition of Machine Learning approaches: concepts of learning from data, supervised vs. unsupervised learning, feature extraction, hypothesis presentation.
    \item Definition of fields of use through concrete examples: prediction, classification, verification, clustering.
\end{itemize}
Part II - Machine learning algorithms: theory and applications (60\%):
\begin{itemize}
    \item Introduction au framework KNIME
    \item Raw data to useful features: preprocessing algorithms and feature extraction
    \item Clustering
    \item Rules of association
    \item Approches Bayesiennes
    \item Decision trees
    \item Artificial neural networks
    \item Gaussian Mixtures
\end{itemize}
Part III - Advanced applications (20\%):
\begin{itemize}
    \item Time signal processing
    \item State-based modeling
    \item Heterogeneous data processing, merging principles
\end{itemize}