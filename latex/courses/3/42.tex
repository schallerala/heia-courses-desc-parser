% id: 42
\gradedCourse{Elective: NoSQL}{2}{5.1}{Chapitre spécialisé: Introduction au monde NoSQL}{french}{Houda Chabbi, Benoit Perroud}
Since about ten years the NoSQL and bigdata movement has been growing more and more. Initially presented as the magical alternative to the immutable RDBMSs, the current trend is moderating these words to insist on the complementarity of all these technologies with each other. The future, in terms of data storage, will therefore take the form of "polyglot persistence". For designers and developers of new information systems, it is therefore a question of being able to set up an architecture that mixes all these technologies to good effect. To do this, this chapter proposes to understand what these technologies known as NoSQL are, to see their advantages and disadvantages in order to allow a judicious use of them. The following concepts will be discussed:
\begin{itemize}
    \item Scale-out vs. scale-in and BASE vs. ACID properties
    \item Distributed storage and distributed computing (HDFS, Map Reduce, TEZ, Spark...)
    \item Key-value warehouses with the following demo tool: Redis
    \item The document-oriented databases with as demonstration tool: MongoDB
    \item Column oriented databases with Cassandra as demonstration tool
    \item Graph oriented databases with Neo4j as a demonstration tool.
\end{itemize}
This course will be given by Houda Chabbi Drissi (HEIAFR) and Benoit Perroud (Verisign).