% id: 48
\gradedCourse{Human-Computer Interaction 2}{2}{5.1}{Interfaces homme-machine 2}{french}{Sandy Ingram}
Main content (technical and ergonomic aspects):
\begin{itemize}
    \item Principles and techniques of User-Centered Design (UCD).
    \item Main components of UX (user experience): utility, usability, and emotional impact.
    \item Application of classical engineering techniques in the development cycle of user interfaces and interactive systems.
    \item Methods and techniques specific to each phase of the development cycle (e.g. ideation, affinity diagram, personas, wireframe, interactive mockup).
    \item Ergonomic principles and criteria for the development of 'usable' interfaces.
    \item Introduction to 'Material Design' (its link with ergonomic criteria, its advantages over flat design).
    \item Introduction to the FLUX interface design pattern (specificities, advantages, links with other design patterns and MVC variants).
    \item Introduction to the REACT library (based on FLUX and used in the development of front-end user interfaces).
    \item Types of user interface evaluation (summative vs. formative, empirical vs. analytical, based on qualitative vs. quantitative data, rapid vs. rigorous).
    \item Standard questionnaires used in empirical evaluations.
    \item Adaptation of the evaluation to the development phase and context of the project.
\end{itemize}
Optional content (depending on the chosen themes and the time available):
\begin{itemize}
    \item Introduction to the REDUX bookstore.
\end{itemize}