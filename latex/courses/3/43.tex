% id: 43
\gradedCourse{Elective: Game Design and Development}{2}{5.5}{Chapitre spécialisé: Game Design and Development}{french}{Maurizio Rigamonti, Maurizio Caon}
The course consists of a theoretical part and a practical part (directed work with a tutorial to learn how to use Unity and to make a video game prototype). The theoretical part includes :
\begin{itemize}
    \item Introduction to video game design: presentation and explanation of design methods, collaboration in a multidisciplinary team and interaction with end users.
    \item Presentation and explanations of formal elements, with focus on game mechanics (e.g. game rules, definition of actions, how to manage the user experience, etc.) and their balancing, interface structure and interaction modalities.
    \item Presentation of dramatic elements such as story, scenarios, characters and graphic styles.
    \item Introduction to 'Gamification' and 'Serious Game' practices.
    \item Introduction to starting a business, industry business models, and fundraising. Presentation of career opportunities in the field.
\end{itemize}
The practical teaching will be divided into two parts: a tutorial on how Unity works and a project done by groups of 2 or 3 students. The project will consist in imagining, designing and developing a video game without constraints of the technologies to be used (students will have at their disposal several peripherals to be used for the development of the interface of the video game, for example: Microsoft Kinect, Wii remote controller, Oculus Rift et cetera).