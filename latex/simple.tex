\documentclass[11pt]{article}

\usepackage[
    a4paper,
    hmargin=2cm, vmargin=1.3cm,
    headheight=30pt, % as per the warning by fancyhdr
    includehead, includefoot
]{geometry}

\usepackage[english]{babel}
\usepackage[utf8]{inputenc}

\usepackage{fancyhdr} % extensice control of page headers and footers
\usepackage{graphicx}
\usepackage{parskip} % add space between paragraphs and remove first line's indent
\usepackage{ifthen}
\usepackage{titlesec}
\usepackage{hyperref}
\usepackage[dvipsnames]{xcolor} % color the text for the color code: https://en.wikibooks.org/wiki/LaTeX/Colors

% ---------------- SETUP HEADER & FOOTER ----------------------
\pagestyle{fancy}
\fancyhf{} % sets both header and footer to nothing
\renewcommand{\headrulewidth}{0pt} % remove headers bottom line

\hypersetup{
	colorlinks = false,
	% linkcolor = red,
	% menucolor = red,
	% filecolor = blue,
	% anchorcolor = green,
	% urlcolor = blue,
	% allcolors = MidnightBlue,
	linkbordercolor = white
}

\addto\captionsenglish{% Replace "english" with the language you use
  \renewcommand{\contentsname}%
    {Summary}%
}
\setcounter{tocdepth}{2}

\newcommand*{\heia}{School of Engineering and Architecture of Fribourg}
\newcommand{\module}[2]{
    \subsection[#1]{#1 \small -- $#2$ ECTS}
}
% #1: English name
% #2: Credits
% #3: Grade
% #4: Orig name
% #5: Taught lang
% #6: Teacher
\newcommand{\gradedCourse}[6]{
    \subsubsection{#1 \small -- $#2$ ECTS -- Swiss grade $#3 / 6$\\[0.4em]\footnotesize{#4 in #5 given by #6}}
}
% #1: English name
% #2: Credits
% #3: Orig name
% #4: Taught lang
% #5: Teacher
\newcommand{\validatedCourse}[5]{
    \subsubsection{#1 \small -- $#2$ ECTS -- \textit{Validated}\\[0.4em]\footnotesize{#3 in #4 given by #5}}
}

\lhead{Alain Schaller\\Aalto University\\Master application 2020}
\rhead{Courses Descriptions\\\heia{}\\Switzerland}
\rfoot{Page \thepage}

\let\tempone\itemize
\let\temptwo\enditemize
\renewenvironment{itemize}{\tempone\addtolength{\itemsep}{-0.3\baselineskip}}{\temptwo}

%                          left        before          after
\titlespacing*{\subsection}{0pt}{3\baselineskip}{0.8\baselineskip}


% ---------------------- DOCUMENT ----------------------------
\begin{document}

\textsc{\LARGE Relevant Courses Descriptions taken at}%\\%[0.2em]
\begin{center}
    \textbf{\heia{}, Switzerland\\Bachelor}
\end{center}

\tableofcontents

%
%
% ---------------------------------
\section{1$^{st}$ year}
\module{Mathematics}{14}
% id: 12
\course{Linear Algebra 1}{Algèbre linéaire 1}{German}{3.5}{4.7}{Mathematics}{TODO} % Algèbre linéaire 1
% id: 14
\course{Linear Algebra 2}{Algèbre linéaire 2}{German}{3.5}{4.8}{Mathematics}{TODO} % Algèbre linéaire 2
% id: 13
\gradedCourse{Calculus 1}{3.5}{4.6}{Analyse 1}{german}{Micha~Wasem}
Functions, limits and continuity:
\begin{itemize}
    \item Functions (algebraic, trigonometric, transcendental, piecewise, ...)
    \item Composition of functions
    \item Reciprocal functions
    \item Intuitive definition of the limit and continuity of a function
    \item Properties
\end{itemize}
Derivatives and applications:
\begin{itemize}
    \item Slope of a tangent and definition of the derivative
    \item Derivation techniques (rules, chain derivation, implicit, ...)
    \item Extreme values of a continuous function, growth and concavity
    \item Optimization and related rates
    \item Differential and linear approximation
    \item Hospital Rule
\end{itemize} % Analyse 1
\pagebreak
% id: 15
\gradedCourse{Calculus 2}{3.5}{4.9}{Analyse 2}{german}{Micha~Wasem}
Complete:
\begin{itemize}
    \item Primitives of a function and indefinite integral
    \item Riemann sum and defined integral
    \item Fundamental theorem and integration techniques
    \item Introduction to differential equations
    \item Applications: calculation of areas in the plane, lengths, volumes and areas of revolution, concepts of work.
    \item Improper integrals
\end{itemize}
Infinite series:
\begin{itemize}
    \item Consequences and their limits
    \item Definitions and convergence criteria for the series
    \item Taylor Series
\end{itemize} % Analyse 2

\module{Basic computing Science}{17}
% id: 1
\gradedCourse{Programming}{6.5}{5.1}{Programmierung}{german}{Jacques Supcik}
Introduction to Programming:
\begin{itemize}
    \item Concept of computer (architecture, main components) and program
    \item Development: steps, tools used, files involved, work environment
    \item Mastering the basic functions of a development environment (editing, compiling, debugging)
    \item Concepts of type and variable; distinction between primitive type and reference type
    \item Instructional and Program Concepts
\end{itemize}
Programming in Java:
\begin{itemize}
    \item Language basics, identifiers, reserved words, variables, primitive types, expressions and assignments
    \item Control structures: sequence, selection (if..else and switch instructions)
    \item Instructions itératives (boucles while, do..while et for)
    \item Methods and functions (specification, implementation, use); parameter passing
    \item One- and multi-dimensional tables
    \item Creating and manipulating strings (String)
    \item Exception generation, handling and propagation
    \item Simple inputs/outputs (I/O), flow concepts, reading and writing text files
    \item Classes and objects
    \item Packages and access control
    \item Static members (fields and methods)
    \item Concept of inheritance and creation of subclasses
    \item Abstract classes and interfaces
    \item Classes internes, anonymes
    \item Expressions lambdas
    \item General
    \item Stream (flow-based programming)
\end{itemize} % Programmierung
% id: 2
\gradedCourse{Algorithmics and Data Structures}{6.5}{5.3}{Algorithmen}{german}{Andreas Fischer}
Course content included:
\begin{itemize}
    \item Notion of algorithm and complexity analysis.
    \item Notion of abstract type, case studies (stack, queues, chained list, set, hash tables, bitsets, priority queue, introduction to trees and graphs).
    \item Programming techniques (recursivity, chaining, dynamic programming, error processing, genericity, probabilistic algorithms, examples of backtracking).
    \item Some classical algorithms (sorting, binary search, random numbers, hash, RSA cryptography...).
    \item Problem-solving approach, program specification by PRE/POST-conditions, program testing, systematic fight against bugs, numerical problems in programming, measurement of execution time
\end{itemize} % Algorithmen
% id: 3
\gradedCourse{Human-Computer Interaction}{4}{5}{Interface Homme-Machine 1}{french}{Sandy Ingram}
Course content included:
\begin{itemize}
    \item Putting into practice the elements seen during programming: classes, abstract classes, interfaces, internal classes, enum, genericity, annotations, lambda expressions, method references, etc.
    \item Technical aspects related to the use of JavaFX (Stage, scene graphs, containers, UI components, properties, binding, event management, ...). Principles of event-driven programming.
    \item Using the SceneBuilder tool to generate graphical interfaces in declarative mode (XML / FXML format).
    \item Splitting of an application according to the Model-View-Controller (MVC) architecture.
    \item Drop-down and context-sensitive menus. Creation of keyboard shortcuts.
    \item Creation and use of general and specialized dialog boxes.
    \item Deployment of an application with external resources (images, sound, ...).
    \item General principles related to human-machine interaction, usability, ergonomic criteria and human factors.
    \item Ergonomic assessments: techniques and implementation.
    \item Elements related to visualization (icons, highlighting techniques, layout and alignment, screen template, gaze path, layout, use of colors, definitions and typographical rules, ...).
\end{itemize} % Interface Homme-Machine 1

\pagebreak
\module{Discrete Electronics}{8}
% id: 16
\gradedCourse{Discrete Electronics 1}{4}{5.1}{Numerische Technik 1}{german}{Michael~Mäder}
Block TN1: Numbering, Combinatorial Systems:
\begin{itemize}
    \item Binary numbers and arithmetic operations
    \item Logical operations and functions.
    \item Analysis and synthesis of combinatorial functions.
    \item Introduction to VHDL for combinatorial circuits.
\end{itemize}
Block TN2: Simple synchronous sequential systems:
\begin{itemize}
    \item Analysis of sequential systems
    \item Synthesis of synchronous sequential systems
\end{itemize}
Practical work:
\begin{itemize}
    \item Introduction to the laboratory and standard CMOS circuits
    \item Combinatorial circuit
    \item Physical synchronous circuit
    \item Technology, propagation time
    \item Use of an oscilloscope: measurement of voltages, time.
    \item Impedance matching: display of different voltages and signal propagation times.
\end{itemize} % Numerische Technik 1
% id: 17
\gradedCourse{Discrete Electronics 2}{4}{4.7}{Numerische Technik 2}{german}{Michael Mäder}
Block TN3: Advanced synchronous sequential systems:
\begin{itemize}
    \item Analysis and synthesis of advanced sequential systems
\end{itemize}
Block TN4: Special systems and technology:
\begin{itemize}
    \item Analysis and Synthesis of Sequential Asynchronous Systems.
    \item CMOS logic circuit technology.
    \item Arithmetic systems
\end{itemize}
Block TN5: Programmable Systems:
\begin{itemize}
    \item Programmable circuits.
    \item Introduction to the VHDL description.
\end{itemize}
Practical work:
\begin{itemize}
    \item Synchronous circuit in simulation
    \item Physical synchronous circuit
    \item Programming an FPGA
\end{itemize}
Integrated project, carried out in a group:
\begin{itemize}
    \item Complete management of a group project (sessions, specifications/operating instructions, design, production, testing, documentation and presentation)
\end{itemize} % Numerische Technik 2

\module{Networking}{8}
% id: 18
\course{Networking 1}{Téléinformatique 1}{German}{4}{4.2}{Networking}{TODO} % Téléinformatique 1
% id: 19
\gradedCourse{Networking 2}{4}{5.4}{Téléinformatique 2}{german}{Rudolf Scheurer}
Course content included:
\begin{itemize}
    \item Protocols and mechanisms of the data link, network and transport layers
    \item Local area network technologies: Ethernet/TCP/IP protocol stack
    \item Addressing: IPv4, classes, private addressing, introduction to IPv6
    \item Network elements: hubs, bridges, switches, routers, address servers, etc.
    \item Application protocols: e-mail, file transfer, virtual terminals, HTTP, etc.
\end{itemize}
Practical work:
\begin{itemize}
    \item Data Layer Protocol (PPPoE)
    \item MAC/IP addressing (ARP)
    \item Internet: IP, ICMP, fragmentation
    \item Internet: UDP, TCP
    \item Internet applications: Telnet, HTTP, FTP, SMTP
    \item Address management with DHCP
\end{itemize} % Téléinformatique 2

\pagebreak
\module{Languages, communication and management}{13}
% % id: 4
\course{Technical German 1}{Allemand technique 1}{German}{1.5}{Validated}{Languages, communication and management}{TODO} % Allemand technique 1
% % id: 9
\course{Technical German 2}{Allemand technique 2}{German}{1.5}{Validated}{Languages, communication and management}{TODO} % Allemand technique 2
% id: 5
\course{English 1}{Anglais 1}{English}{1.5}{5.9}{Languages, communication and management}{TODO} % Anglais 1
% id: 10
\gradedCourse{English 2}{1.5}{5.3}{Anglais 2}{english}{Santiago Cruz}
Continuation and development of all activities defined in Module 1, but in addition:
\begin{itemize}
    \item business English knowledge, such as writing appropriate application (cover) letters and CVs (résumés), according not only to normal writing practice in the USA and UK, but also to Swiss and international regulations (both academic and corporate)
\end{itemize} % Anglais 2
% % id: 6
\gradedCourse{Communcation 1}{1.5}{5.1}{Communication 1}{german}{Bruno Müller}
Public Speaking:
\begin{itemize}
    \item Definition of communication
    \item Communication impact and the importance of non-verbal communication
    \item The target audience: Who are we talking about?
    \item Define the objectives of the communication
    \item Manage speaking time
    \item Communication ambiguity, brakes and interference
    \item Advantages and disadvantages of visual aids
    \item Designing and using visual aids
    \item Body language and its importance in oral communication
\end{itemize} % Communication 1
% % id: 11
\gradedCourse{Communication 2}{1.5}{5.3}{Communication 2}{german}{Bruno~Müller}
Public Speaking:
\begin{itemize}
    \item Putting your thoughts in order
    \item The written presentation of a document
    \item Group work
    \item The cover letter and the job interview
    \item The company's information channels
    \item Advertising strategy
\end{itemize} % Communication 2
% % id: 7
\course{Methodology}{Methodologie}{German}{2}{4.2}{Languages, communication and management}{TODO} % Methodologie
% % id: 8
\validatedCourse{IT Project}{2}{Projet TIC}{french}{François Buntschu, François Kilchoer, Jean-Roland Schuler
Group projects in the fields of robotics and computer and network security: % Projet TIC


%
%
% ---------------------------------
\pagebreak
\section{2$^{nd}$ year}
\module{Mathematics and Sciences for Computer Scientists}{12}
% id: 24
\gradedCourse{Specialized Mathematics 1}{3.5}{5}{Mathématiques spécifiques 1}{french}{Roseline Nussbaumer}
Machine learning:
\begin{itemize}
    \item multivariate functions
    \item gradient
    \item optimization (under constraints)
    \item gradient descent
    \item ...
\end{itemize}
Discrete Mathematics:
\begin{itemize}
    \item equity relationships
    \item generating sets
    \item linear recurrences
    \item first-order logic
\end{itemize} % Mathématiques spécifiques 1
% id: 26
\gradedCourse{Specialized Mathematics 2}{3}{5.4}{Mathématiques spécifiques 2}{french}{Rudolf~Riedi}
Discrete Mathematics:
\begin{itemize}
    \item Introduction to number theory (modules, primality tests etc.).
    \item Basics of Shannon's information theory.
    \item Bases de la cryptographie moderne ('public key', 'zero knowledge' etc.).
\end{itemize}
Digital Mathematics:
\begin{itemize}
    \item Technique for searching zeros of non-linear functions
    \item Simplex algorithm
    \item Unavoidable errors on the computer (floating point numbers ...)
    \item Polynomial, linear, spline and Bézier curve interpolations
\end{itemize}
Formal aspects of languages:
\begin{itemize}
    \item Notion of alphabet, language, operators on languages (union, intersection, concatenation, Kleene closure);
    \item Deterministic finite state automata, regular languages, regular expressions;
    \item Syntax diagrams, EBNF notation, derivation, terminal and non-terminal symbols;
    \item Lexical/syntactic/semantic analysis, language interpretation, recursive descent algorithm, arithmetic expression analysis;
    \item Formal program properties, program proofs, assertions, invariants, pre/post conditions, JML, propositional and temporal logic;
    \item Turing machines, undecidable problems, classes P, NP, and NP-complete.
\end{itemize} % Mathématiques spécifiques 2
% id: 27
\gradedCourse{Physics SIAM}{4}{4.4}{Physique SIAM}{french}{Ales~Janka}
Kinematics:
\begin{itemize}
    \item Speed: constant; instantaneous - derivative; average; relative motion
    \item Acceleration: medium; instantaneous
    \item Free fall and oblique jet
    \item MRU and MRUA
\end{itemize}
Forces:
\begin{itemize}
    \item 3 Newton's laws: law of inertia; force, mass, quantity of motion; action-reaction.
    \item Force Dynamics
    \item Gravitational force
\end{itemize}
Energy:
\begin{itemize}
    \item Notion of work
    \item Power
    \item Mechanical energy: kinetic, potential
\end{itemize}
Oscillations, waves:
\begin{itemize}
    \item Notion of electromagnetic wave
    \item Frequency
    \item Phase
    \item Light, sound; speed of propagation
\end{itemize}
Electricity (learning reminders):
\begin{itemize}
    \item Ohm's law; power; basic circuits; parallel and series resistors
\end{itemize}
Possibly:
\begin{itemize}
    \item Solids Statics
    \item Fluids: Archimedes, Pascal
\end{itemize} % Physique SIAM
% id: 25
\course{Statistics}{Statistiques}{German}{1.5}{4.5}{Mathematics and Sciences for Computer Scientists}{TODO} % Statistiques

\module{Project and Project Management}{6}
% % id: 28
\gradedCourse{Business Management}{1.5}{5.3}{Betriebswirtschaft}{german}{Alfred~Münger}
Course content included:
\begin{itemize}
    \item The company and its environment (general conditions, structure, functions, relationships, legal forms etc.)
    \item Financial aspects (overall objectives, investment, financing)
    \item Accounting aspects (bookkeeping, balance sheet, ER, analysis)
    \item Corporate management (management functions, HR)
    \item Management techniques (analysis methods, scenarios)
\end{itemize} % Betriebswirtschaft
% id: 29
\course{Project Management ICT}{Projektmanagement ICT}{German}{2}{5.2}{Project and Project Management}{TODO} % Projektmanagement ICT
\pagebreak
% id: 30
\gradedCourse{Semester Project}{2.5}{5.7}{Projet de semestre}{french}{Frédéric Bapst, Sandy Ingram, François Kilchoer}
For the semester project, teachers form groups of 3-6 people. The project generally involves the computer production of a game (e.g. jass, dots-and-boxes, naval battle...). All groups follow a common specification, which allows the modules to be exchanged and the results to be compared (e.g. the strategies of the automatic player). The subject allows to develop the playful, creative and competitive aspects. In addition to the two weekly scheduled periods, there are at least two other work periods taken from the student's free time.: % Projet de semestre

\module{Algorithmic, Database, Software Engineering}{17}
% id: 20
\gradedCourse{Data Structures and Algorithms 2 \& Labs}{3.5}{5.5}{Algorithmique et structures de données 2}{french}{Frédéric~Bapst}
Course content included:
\begin{itemize}
    \item Damped complexity, expected complexity.
    \item Arbre, parcours, heap, skew-heap, arbre de fouille (binary search tree), arbres-splay, treap, B-arbres.
    \item Disjoint sets, discrete simulation, string searching, editing distance, compression (Huffman, LempelZiv)
    \item Pseudo-pointers, generality, some clever algorithms, tests and programmer's tools
\end{itemize} % Algorithmique et structures de données 2
% id: 23
\gradedCourse{Data Structures and Algorithms 3 \& Labs}{5.5}{5}{Algorithmique et structures de données 3}{french}{Frédéric~Bapst}
Course content included:
\begin{itemize}
    \item Graph, path, shortest path algorithms, flows and breaks in a network, connectivity, overlay tree, topological sorting. Some other concepts (click, matching, coloring, bipartite, condensation...).
    \item Backtracking, alpha/beta
    \item Computational geometry: primitives, direction of rotation, scanning technique, points in a polygon, polygon map, RangeTree, IntervalTree, duality, clustering.
    \item Code optimization. Java bytecode (concepts and tools)
\end{itemize} % Algorithmique et structures de données 3
% id: 22
\gradedCourse{Software Engineering 1}{4}{5.4}{Génie Logiciel 1}{french}{Pierre~Kuonen and Elena~Mugellini}
Course content included:
\begin{itemize}
    \item Software Engineering Motivation
    \item Principle of object-oriented modeling
    \item Main UML diagrams
    \item Using a tool for modeling with UML
\end{itemize} % Génie Logiciel 1
% id: 21
\gradedCourse{Databases 1}{4}{5.3}{Bases de données 1}{french}{Houda Chabbi}
Course content included:
\begin{itemize}
    \item General introduction to DBMS.
    \item Functions of a database management system (logical and physical independence, data consistency, data sharability, non-redundancy of data, data security, efficiency of data access).
    \item Entity-association data model.
    \item Relational model and relational algebra
    \item Transition from the entity-association model to the relational model.
    \item Language for defining, manipulating and controlling SQL data.
    \item Extended SQL: trigger, stored procedure
    \item Dynamic SQL.
    \item Understanding and avoiding SQL injection
\end{itemize} % Bases de données 1

\module{Information Systems and Mobile Applications}{14}
% id: 34
\gradedCourse{Information Systems 1}{2}{5}{Systèmes d'information 1}{french}{Elena Mugellini}
Document modeling \& representing:
\begin{itemize}
    \item Document Model
    \item XML Language (syntax, tag)
    \item DTD (Document Type Definition)
    \item XML Schema - basic concepts
\end{itemize}
Transforming Document:
\begin{itemize}
    \item XPATH Language
    \item XSLT Language
\end{itemize}
Document Rendering:
\begin{itemize}
    \item CSS (Cascading Style Sheet)
    \item XSL-FO (Formatting Object) Language
    \item SVG (Scalable Vector Graphics), SMIL (Synchronized Multimedia Integration Language), etc.
\end{itemize}
Programming Document:
\begin{itemize}
    \item DOM (Document Object Model) standard
    \item SAX (Simple API for XML) standard
    \item JDOM (Java API for DOM), JAXP (Java API for XML Processing)
\end{itemize} % Systèmes d'information 1
\pagebreak
% id: 31
\course{Concurrent Programming 1}{Programmation concurrente 1}{French}{4}{5.2}{Information Systems and Mobile Applications}{TODO} % Programmation concurrente 1
% id: 33
\gradedCourse{Concurrent Programming 2}{4}{5.1}{Programmation concurrente 2}{french}{François~Kilchoer}
Course content included:
\begin{itemize}
    \item Introduction: first approach, definitions, characteristics and model.
    \item Communication: issues, implementation (asynchronous and synchronous messages, appointments, CPR and RMI)
    \item Development languages/environments that support distributed programming: JR, Java, etc.
    \item Distributed programming.
    \item Parallel programming.
\end{itemize} % Programmation concurrente 2
% id: 32
\gradedCourse{Mobile Applications 1}{4}{5.1}{Applications Mobiles 1}{french}{Pascal Bruegger}
Through this course the student will become familiar with Android mobile concepts:
Major chapters:
\begin{itemize}
    \item Study of the SDK and the components of the development platform.
    \item Study of the main components: Activity and service, Intent and broadcast receiver, Fragment, Persistence and Content Provider
    \item Programming of applications as TP and ED
    \item Application deployment on smartphone and tablet
\end{itemize} % Applications Mobiles 1

\pagebreak
\module{Embedded systems and operating systems}{11}
% id: 36
\gradedCourse{Embedded-Systems 1}{4}{5.1}{Embedded-Systems 1}{german}{Jacques Supcik}
The course will cover the following topics:
\begin{itemize}
    \item The C language and its use (basic types and declarations, complex types, functions, pointers and function pointers)
    \item Object-oriented programming in C
    \item Development tools (generation, validation, documentation)
    \item ARM microprocessor instruction set (assembler)
    \item ARM microprocessor addressing modes
    \item Assembler Interface - C
    \item Floating point and integer processing
    \item Implementation of simple peripherals
\end{itemize} % Embedded-Systems 1
% id: 35
\gradedCourse{Operating Systems 1}{3}{5.1}{Système d'exploitation 1}{french}{François Kilchoer}
Course content included:
\begin{itemize}
    \item Memory, organization and management: main memory, virtual memory, management algorithms.
    \item Scheduling algorithms - basic mechanisms and examples
    \item Disk, organization and management: performance optimization, file management systems, Unix/NT examples, security, protection and backup.
    \item Input-output: basic mechanisms, Unix/NT examples.
    \item Computer tools/shell programming: sh and main variants; common commands (coreutils) + some tools for experts (grep, sed, awk, python).
\end{itemize} % Système d'exploitation 1
% id: 37
\gradedCourse{Embedded-Systems 2}{4}{5}{Systèmes Embarqués 2}{french}{Jacques Supcik}
The course will cover the following topics:
\begin{itemize}
    \item General microprocessor architecture (Von Neuman, Harvard, SIMD, MIMD)
    \item Internal microprocessor architecture
    \item Programmation interruptive
    \item Interconnection of common peripherals
    \item Operating system kernel elements (thread, semaphore, etc.)
    \item Direct Memory Access (DMA)
    \item Hidden memories
    \item Memory Management Unit (MMU)
\end{itemize} % Systèmes Embarqués 2


%
%
% ---------------------------------
\section{3$^{rd}$ year}
\module{Advanced Computer Science}{8}
% id: 43
\gradedCourse{Elective: Game Design and Development}{2}{5.5}{Chapitre spécialisé: Game Design and Development}{french}{Maurizio Rigamonti, Maurizio Caon}
The course consists of a theoretical part and a practical part (directed work with a tutorial to learn how to use Unity and to make a video game prototype) The theoretical part includes :
\begin{itemize}
    \item Introduction to video game design: presentation and explanation of design methods, collaboration in a multidisciplinary team and interaction with end users.
    \item Presentation and explanations of formal elements, with focus on game mechanics (e.g. game rules, definition of actions, how to manage the user experience, etc.) and their balancing, interface structure and interaction modalities.
    \item Presentation of dramatic elements such as story, scenarios, characters and graphic styles.
    \item Introduction to 'Gamification' and 'Serious Game' practices.
    \item Introduction to starting a business, industry business models, and fundraising. Presentation of career opportunities in the field.
\end{itemize}
The practical teaching will be divided into two parts: a tutorial on how Unity works and a project done by groups of 2 or 3 students. The project will consist in imagining, designing and developing a video game without constraints of the technologies to be used (students will have at their disposal several peripherals to be used for the development of the interface of the video game, for example: Microsoft Kinect, Wii remote controller, Oculus Rift et cetera).:
This course will be given by : Maurizio Caon, Maurizio Rigamonti \& Omar Abou Khaled: % Chapitre spécialisé: Game Design and Development
% id: 42
\gradedCourse{Elective: NoSQL}{2}{5.1}{Chapitre spécialisé: Introduction au monde NoSQL}{french}{Houda~Chabbi and Benoit~Perroud}
Since about ten years the NoSQL and bigdata movement has been growing more and more. Initially presented as the magical alternative to the immutable RDBMSs, the current trend is moderating these words to insist on the complementarity of all these technologies with each other. The future, in terms of data storage, will therefore take the form of "polyglot persistence". For designers and developers of new information systems, it is therefore a question of being able to set up an architecture that mixes all these technologies to good effect. To do this, this chapter proposes to understand what these technologies known as NoSQL are, to see their advantages and disadvantages in order to allow a judicious use of them. The following concepts will be discussed:
\begin{itemize}
    \item Scale-out vs. scale-in and BASE vs. ACID properties
    \item Distributed storage and distributed computing (HDFS, Map Reduce, TEZ, Spark...)
    \item Key-value warehouses with the following demo tool: Redis
    \item The document-oriented databases with as demonstration tool: MongoDB
    \item Column oriented databases with Cassandra as demonstration tool
    \item Graph oriented databases with Neo4j as a demonstration tool.
\end{itemize} % Chapitre spécialisé: Introduction au monde NoSQL
\pagebreak
% id: 45
\gradedCourse{Elective: Macine Learning}{2}{4.9}{Chapitre spécialisé: Machine Learning Applications}{french}{Elena~Mugellini and Jean~Hennebert}
Introduction:
\begin{itemize}
    \item Machine Learning is a branch of Artificial Intelligence that studies so-called machine learning algorithms. These algorithms are capable, using examples, of solving complex problems that would be difficult to solve using traditional approaches. Machine Learning is used today in several fields: prediction (stock market evolution, weather), classification (gesture recognition, speech recognition, image recognition), verification/detection (biometric authentication), data mining (clustering on complex data).
\end{itemize}
Method:
\begin{itemize}
    \item The course combines theoretical and practical sessions. Of particular importance is given to practical sessions, which will be done through guided classroom exercises (including the use of the KNIME and WEKA tools) and through mini-projects carried out in groups of 2-3 people.
\end{itemize}
Part I - Basic principles (20\%):
\begin{itemize}
    \item Definition of Machine Learning approaches: concepts of learning from data, supervised vs. unsupervised learning, feature extraction, hypothesis presentation.
    \item Definition of fields of use through concrete examples: prediction, classification, verification, clustering.
\end{itemize}
Part II - Machine learning algorithms: theory and applications (60\%):
\begin{itemize}
    \item Introduction au framework KNIME
    \item Raw data to useful features: preprocessing algorithms and feature extraction
    \item Clustering
    \item Rules of association
    \item Approches Bayesiennes
    \item Decision trees
    \item Artificial neural networks
    \item Gaussian Mixtures
\end{itemize}
Part III - Advanced applications (20\%):
\begin{itemize}
    \item Time signal processing
    \item State-based modeling
    \item Heterogeneous data processing, merging principles
\end{itemize} % Chapitre spécialisé: Machine Learning Applications
\pagebreak
% id: 44
\course{Elective: IT startup bootcamp}{Chapitre spécialisé: IT startup bootcamp}{French}{2}{5.5}{Advanced Computer Science}{TODO} % Chapitre spécialisé: IT startup bootcamp

\module{Advanced Computer Science 2}{10}
% id: 48
\gradedCourse{Human-Computer Interaction 2}{2}{5.1}{Interfaces homme-machine 2}{french}{Sandy Ingram}
Main content (technical and ergonomic aspects):
\begin{itemize}
    \item Principles and techniques of User-Centered Design (UCD).
    \item Main components of UX (user experience): utility, usability, and emotional impact.
    \item Application of classical engineering techniques in the development cycle of user interfaces and interactive systems.
    \item Methods and techniques specific to each phase of the development cycle (e.g. ideation, affinity diagram, personas, wireframe, interactive mockup).
    \item Ergonomic principles and criteria for the development of 'usable' interfaces.
    \item Introduction to 'Material Design' (its link with ergonomic criteria, its advantages over flat design).
    \item Introduction to the FLUX interface design pattern (specificities, advantages, links with other design patterns and MVC variants).
    \item Introduction to the REACT library (based on FLUX and used in the development of front-end user interfaces).
    \item Types of user interface evaluation (summative vs. formative, empirical vs. analytical, based on qualitative vs. quantitative data, rapid vs. rigorous).
    \item Standard questionnaires used in empirical evaluations.
    \item Adaptation of the evaluation to the development phase and context of the project.
\end{itemize}
Optional content (depending on the chosen themes and the time available):
\begin{itemize}
    \item Introduction to the REDUX bookstore.
\end{itemize} % Interfaces homme-machine 2
\pagebreak
% id: 46
\course{Databases 2}{Bases de données 2}{French}{2}{4.6}{Advanced Computer Science 2}{TODO} % Bases de données 2
% id: 47
\gradedCourse{Software Engineering 2}{3}{5}{Génie logiciel 2}{french}{Pierre Kuonen}
Course content included:
\begin{itemize}
    \item OO architecture principles: Component and deployment diagrams.
    \item Main design patterns
    \item Principles of 3-tier architecture
    \item Software components: Design and implementation of a client/server system.
\end{itemize} % Génie logiciel 2
% id: 49
\gradedCourse{Logic Programming}{3}{5.4}{Programmation logique}{french}{Frédéric Bapst}
Logic programming (64 periods):
\begin{itemize}
    \item Basics of Prolog: logic, predicates, logic variable, bypass trees.
    \item Incomplete data structures.
    \item Browse through report networks.
    \item Cutaneous problems, negative. Dynamic rules (assert).
    \item Definite Clause Grammars, use of Prolog for syntax analysis.
    \item Meta-programming. Notions of blackboards.
    \item Extension of the logic model by the expression of constraints.
    \item Propagation/distribution, constraint programming techniques, GnuProlog engine
    \item Programmation multi langages
\end{itemize} % Programmation logique

\pagebreak
\module{Applied Computer Science 1}{9}
% % id: 40
\gradedCourse{Microprocessor Systems}{2}{3.8}{Microprocesseurs}{french}{Jean-Roland~Schuler}
Course content included:
\begin{itemize}
    \item InstructionSet of processors. gcc options related to theSet statement
    \item Organization of the cache memory. Programming in C to make optimal use of the cache memory
    \item Organization of pipelining. Programming in C for optimal use of pipelining
    \item Use of Valgrind to control-optimize a program.
    \item Intel processor architectures
    \item Reverse Engineering d'un executable
    \item Basics of GPU programming
\end{itemize} % Microprocesseurs
% id: 38
\course{Physics and Simulation}{Physique et simulation}{French}{3}{5.4}{Applied Computer Science 1}{TODO} % Physique et simulation
% % id: 39
\gradedCourse{C/C++ Programming}{2}{4.1}{Programmation C/C++}{french}{François~Kilchoer}
Advanced programming in C/C++ (32 periods):
\begin{itemize}
    \item Java and C++: what differences
    \item The specificities of C
    \item Memory management: pointers, ...
    \item Data structures: Unions, structures, etc...
    \item Classes in C++.
    \item Methods and parameter passing
    \item The operators
    \item Les classes ''Templates''
    \item Multiple inheritance
    \item Compilation: Pre-processor, portability, conditional compilation, ...
    \item Development Tools
\end{itemize} % Programmation C/C++
% id: 41
\course{Web Application Security}{Sécurité des applications}{French}{2}{4.6}{Applied Computer Science 1}{TODO} % Sécurité des applications

\module{Information Systems Specialization}{12}
% id: 52
\gradedCourse{Advanced Java Programming}{2}{5}{Programmation avancée Java}{french}{Rudolf~Scheurer}
Advanced programming in Java:
\begin{itemize}
    \item Basic mechanisms of network communication in Java: Sockets
    \item Interfacing to native software components: JNI (Java Native Interface)
    \item Abstraction layer for remote method invocation: RMI (Remote Method Invocation)
    \item Remote application management with JMX (Java Management eXtension)
\end{itemize} % Programmation avancée Java
% id: 54
\course{Information Systems Architecture}{Architecture des systèmes d'information}{French}{7}{5.2}{Information Systems Specialization}{TODO} % Architecture des systèmes d'information
% id: 53
\gradedCourse{Information Systems 2}{3}{5.8}{Systèmes d'information 2}{french}{Omar~Abou~Khaled and Elena~Mugellini}
Course content included:
\begin{itemize}
    \item Information system (Definition, principles, models, installation and configuration, access protocols, comparison, etc.).
    \item Protocols and languages: HTTP, MIMETYPE, URI, XHTML, CSS.
    \item Advanced XML schema modeling techniques.
    \item Client/server architectures: 2-tier, 3-tier, n-tier and SOA (Services Oriented Architecture) architectures.
    \item Frameworks, application servers and development platforms.
    \item The .NET platform: introduction and operating principle.
    \item NET platform: ASP.NET and ADO.NET.
    \item The Java EE platform: introduction and operating principles.
    \item Java EE: servlets, JSP and JSF.
    \item Overview of the PhP and Ruby on Rails platforms.
    \item Web Services: principles and protocols (UDDI, WSDL, SOAP).
    \item XML databases.
    \item RIA (Rich Internet Application) - AJAX - declarative interfaces (XAML, XUL, etc.).
    \item Directories and meta-directories.
\end{itemize} % Systèmes d'information 2

\pagebreak
\module{Semester Projects}{9}
% id: 50
\gradedCourse{Semester Project 5}{4}{4.8}{Semesterprojekt 5}{german}{Daniel Gachet, Rudolf Scheurer}
Course content included:
\begin{itemize}
    \item Individual or pair projects usually proposed by an external partner (industry, service company, academic institute).
\end{itemize}
Pre-requisite: all successful 2nd year modules: % Semesterprojekt 5
% id: 51
\gradedCourse{Semester Project 6}{5}{5.3}{Semesterprojekt 6}{german}{Omar~Abou~Khaled and Elena~Mugellini}
Course content included:
\begin{itemize}
    \item Individual or two projects that last two blocks, usually proposed by an external partner (industry, service company, academic institute).
\end{itemize} % Semesterprojekt 6

\module{Bachelor Thesis}{12}
% id: 55
\course{Bachelor Thesis}{Travail de Bachelor}{English}{12}{4.9}{Bachelor Thesis}{TODO} % Travail de Bachelor

%
%
% ---------------------------------
\vspace{5em}
\section{Note about the swiss grading system}
\textit{Generally, the following formula is used to calculate a grade in Switzerland:}
\begin{equation}
    Grade = \frac{Points}{MaxPoints} * 5 + 1
\end{equation}

\textit{
A grade of $4$, is the minimum to pass.
}

\end{document}